% Generated by Sphinx.
\def\sphinxdocclass{article}
\documentclass[letterpaper,11pt,openany]{sphinxhowto}
\usepackage[utf8]{inputenc}
\DeclareUnicodeCharacter{00A0}{\nobreakspace}
\usepackage[T1]{fontenc}
\usepackage[english]{babel}
\usepackage{times}
\usepackage[Bjarne]{fncychap}
\usepackage{longtable}
\usepackage{sphinx}
%\documentclass{article}
\setcounter{secnumdepth}{-1}
\renewcommand\tableofcontents{}
\renewcommand\release{}
\renewcommand\author{}


\title{Matthew Brett - curriculum vitae}
\date{October 31, 2012}
\release{}
\author{}
\newcommand{\sphinxlogo}{}
\renewcommand{\releasename}{Release}
\makeindex

\makeatletter
\def\PYG@reset{\let\PYG@it=\relax \let\PYG@bf=\relax%
    \let\PYG@ul=\relax \let\PYG@tc=\relax%
    \let\PYG@bc=\relax \let\PYG@ff=\relax}
\def\PYG@tok#1{\csname PYG@tok@#1\endcsname}
\def\PYG@toks#1+{\ifx\relax#1\empty\else%
    \PYG@tok{#1}\expandafter\PYG@toks\fi}
\def\PYG@do#1{\PYG@bc{\PYG@tc{\PYG@ul{%
    \PYG@it{\PYG@bf{\PYG@ff{#1}}}}}}}
\def\PYG#1#2{\PYG@reset\PYG@toks#1+\relax+\PYG@do{#2}}

\def\PYG@tok@gd{\def\PYG@tc##1{\textcolor[rgb]{0.63,0.00,0.00}{##1}}}
\def\PYG@tok@gu{\let\PYG@bf=\textbf\def\PYG@tc##1{\textcolor[rgb]{0.50,0.00,0.50}{##1}}}
\def\PYG@tok@gt{\def\PYG@tc##1{\textcolor[rgb]{0.00,0.25,0.82}{##1}}}
\def\PYG@tok@gs{\let\PYG@bf=\textbf}
\def\PYG@tok@gr{\def\PYG@tc##1{\textcolor[rgb]{1.00,0.00,0.00}{##1}}}
\def\PYG@tok@cm{\let\PYG@it=\textit\def\PYG@tc##1{\textcolor[rgb]{0.25,0.50,0.56}{##1}}}
\def\PYG@tok@vg{\def\PYG@tc##1{\textcolor[rgb]{0.73,0.38,0.84}{##1}}}
\def\PYG@tok@m{\def\PYG@tc##1{\textcolor[rgb]{0.13,0.50,0.31}{##1}}}
\def\PYG@tok@mh{\def\PYG@tc##1{\textcolor[rgb]{0.13,0.50,0.31}{##1}}}
\def\PYG@tok@cs{\def\PYG@tc##1{\textcolor[rgb]{0.25,0.50,0.56}{##1}}\def\PYG@bc##1{\colorbox[rgb]{1.00,0.94,0.94}{##1}}}
\def\PYG@tok@ge{\let\PYG@it=\textit}
\def\PYG@tok@vc{\def\PYG@tc##1{\textcolor[rgb]{0.73,0.38,0.84}{##1}}}
\def\PYG@tok@il{\def\PYG@tc##1{\textcolor[rgb]{0.13,0.50,0.31}{##1}}}
\def\PYG@tok@go{\def\PYG@tc##1{\textcolor[rgb]{0.19,0.19,0.19}{##1}}}
\def\PYG@tok@cp{\def\PYG@tc##1{\textcolor[rgb]{0.00,0.44,0.13}{##1}}}
\def\PYG@tok@gi{\def\PYG@tc##1{\textcolor[rgb]{0.00,0.63,0.00}{##1}}}
\def\PYG@tok@gh{\let\PYG@bf=\textbf\def\PYG@tc##1{\textcolor[rgb]{0.00,0.00,0.50}{##1}}}
\def\PYG@tok@ni{\let\PYG@bf=\textbf\def\PYG@tc##1{\textcolor[rgb]{0.84,0.33,0.22}{##1}}}
\def\PYG@tok@nl{\let\PYG@bf=\textbf\def\PYG@tc##1{\textcolor[rgb]{0.00,0.13,0.44}{##1}}}
\def\PYG@tok@nn{\let\PYG@bf=\textbf\def\PYG@tc##1{\textcolor[rgb]{0.05,0.52,0.71}{##1}}}
\def\PYG@tok@no{\def\PYG@tc##1{\textcolor[rgb]{0.38,0.68,0.84}{##1}}}
\def\PYG@tok@na{\def\PYG@tc##1{\textcolor[rgb]{0.25,0.44,0.63}{##1}}}
\def\PYG@tok@nb{\def\PYG@tc##1{\textcolor[rgb]{0.00,0.44,0.13}{##1}}}
\def\PYG@tok@nc{\let\PYG@bf=\textbf\def\PYG@tc##1{\textcolor[rgb]{0.05,0.52,0.71}{##1}}}
\def\PYG@tok@nd{\let\PYG@bf=\textbf\def\PYG@tc##1{\textcolor[rgb]{0.33,0.33,0.33}{##1}}}
\def\PYG@tok@ne{\def\PYG@tc##1{\textcolor[rgb]{0.00,0.44,0.13}{##1}}}
\def\PYG@tok@nf{\def\PYG@tc##1{\textcolor[rgb]{0.02,0.16,0.49}{##1}}}
\def\PYG@tok@si{\let\PYG@it=\textit\def\PYG@tc##1{\textcolor[rgb]{0.44,0.63,0.82}{##1}}}
\def\PYG@tok@s2{\def\PYG@tc##1{\textcolor[rgb]{0.25,0.44,0.63}{##1}}}
\def\PYG@tok@vi{\def\PYG@tc##1{\textcolor[rgb]{0.73,0.38,0.84}{##1}}}
\def\PYG@tok@nt{\let\PYG@bf=\textbf\def\PYG@tc##1{\textcolor[rgb]{0.02,0.16,0.45}{##1}}}
\def\PYG@tok@nv{\def\PYG@tc##1{\textcolor[rgb]{0.73,0.38,0.84}{##1}}}
\def\PYG@tok@s1{\def\PYG@tc##1{\textcolor[rgb]{0.25,0.44,0.63}{##1}}}
\def\PYG@tok@gp{\let\PYG@bf=\textbf\def\PYG@tc##1{\textcolor[rgb]{0.78,0.36,0.04}{##1}}}
\def\PYG@tok@sh{\def\PYG@tc##1{\textcolor[rgb]{0.25,0.44,0.63}{##1}}}
\def\PYG@tok@ow{\let\PYG@bf=\textbf\def\PYG@tc##1{\textcolor[rgb]{0.00,0.44,0.13}{##1}}}
\def\PYG@tok@sx{\def\PYG@tc##1{\textcolor[rgb]{0.78,0.36,0.04}{##1}}}
\def\PYG@tok@bp{\def\PYG@tc##1{\textcolor[rgb]{0.00,0.44,0.13}{##1}}}
\def\PYG@tok@c1{\let\PYG@it=\textit\def\PYG@tc##1{\textcolor[rgb]{0.25,0.50,0.56}{##1}}}
\def\PYG@tok@kc{\let\PYG@bf=\textbf\def\PYG@tc##1{\textcolor[rgb]{0.00,0.44,0.13}{##1}}}
\def\PYG@tok@c{\let\PYG@it=\textit\def\PYG@tc##1{\textcolor[rgb]{0.25,0.50,0.56}{##1}}}
\def\PYG@tok@mf{\def\PYG@tc##1{\textcolor[rgb]{0.13,0.50,0.31}{##1}}}
\def\PYG@tok@err{\def\PYG@bc##1{\fcolorbox[rgb]{1.00,0.00,0.00}{1,1,1}{##1}}}
\def\PYG@tok@kd{\let\PYG@bf=\textbf\def\PYG@tc##1{\textcolor[rgb]{0.00,0.44,0.13}{##1}}}
\def\PYG@tok@ss{\def\PYG@tc##1{\textcolor[rgb]{0.32,0.47,0.09}{##1}}}
\def\PYG@tok@sr{\def\PYG@tc##1{\textcolor[rgb]{0.14,0.33,0.53}{##1}}}
\def\PYG@tok@mo{\def\PYG@tc##1{\textcolor[rgb]{0.13,0.50,0.31}{##1}}}
\def\PYG@tok@mi{\def\PYG@tc##1{\textcolor[rgb]{0.13,0.50,0.31}{##1}}}
\def\PYG@tok@kn{\let\PYG@bf=\textbf\def\PYG@tc##1{\textcolor[rgb]{0.00,0.44,0.13}{##1}}}
\def\PYG@tok@o{\def\PYG@tc##1{\textcolor[rgb]{0.40,0.40,0.40}{##1}}}
\def\PYG@tok@kr{\let\PYG@bf=\textbf\def\PYG@tc##1{\textcolor[rgb]{0.00,0.44,0.13}{##1}}}
\def\PYG@tok@s{\def\PYG@tc##1{\textcolor[rgb]{0.25,0.44,0.63}{##1}}}
\def\PYG@tok@kp{\def\PYG@tc##1{\textcolor[rgb]{0.00,0.44,0.13}{##1}}}
\def\PYG@tok@w{\def\PYG@tc##1{\textcolor[rgb]{0.73,0.73,0.73}{##1}}}
\def\PYG@tok@kt{\def\PYG@tc##1{\textcolor[rgb]{0.56,0.13,0.00}{##1}}}
\def\PYG@tok@sc{\def\PYG@tc##1{\textcolor[rgb]{0.25,0.44,0.63}{##1}}}
\def\PYG@tok@sb{\def\PYG@tc##1{\textcolor[rgb]{0.25,0.44,0.63}{##1}}}
\def\PYG@tok@k{\let\PYG@bf=\textbf\def\PYG@tc##1{\textcolor[rgb]{0.00,0.44,0.13}{##1}}}
\def\PYG@tok@se{\let\PYG@bf=\textbf\def\PYG@tc##1{\textcolor[rgb]{0.25,0.44,0.63}{##1}}}
\def\PYG@tok@sd{\let\PYG@it=\textit\def\PYG@tc##1{\textcolor[rgb]{0.25,0.44,0.63}{##1}}}

\def\PYGZbs{\char`\\}
\def\PYGZus{\char`\_}
\def\PYGZob{\char`\{}
\def\PYGZcb{\char`\}}
\def\PYGZca{\char`\^}
% for compatibility with earlier versions
\def\PYGZat{@}
\def\PYGZlb{[}
\def\PYGZrb{]}
\makeatother

\begin{document}

\maketitle
\tableofcontents
\phantomsection\label{research/cv_wrapper::doc}



\section{Personal Details}
\label{research/cv_wrapper:personal-details}\label{research/cv_wrapper:cv}\label{research/cv_wrapper:curriculum-vitae}\begin{description}
\item[{\textbf{Date of birth}}] \leavevmode
November 20, 1964

\item[{\textbf{Nationality}}] \leavevmode
British

\item[{\textbf{email}}] \leavevmode
\href{mailto:matthew.brett@gmail.com}{matthew.brett@gmail.com}

\item[{\textbf{Work address}}] \leavevmode
\begin{DUlineblock}{0em}
\item[] Helen Wills neuroscience institute
\item[] Barker Hall
\item[] University of California
\item[] Berkeley CA 94720
\end{DUlineblock}

\end{description}


\section{Research Positions}
\label{research/cv_wrapper:research-positions}\begin{description}
\item[{\textbf{August 2008 -}}] \leavevmode
Associate researcher at the Brain Imaging Center, University of California, Berkeley

\item[{\textbf{October 2005 – July 2008}}] \leavevmode
Senior investigator scientist at the MRC Cognition \& Brain Sciences Unit
(Cambridge, UK)

\item[{\textbf{October 2003 – September 2005}}] \leavevmode
Associate specialist in psychology at the University of California, Berkeley

\item[{\textbf{March 1999 – September 2003}}] \leavevmode
Research associate at the MRC Cognition \& Brain Sciences Unit (Cambridge, UK)

\item[{\textbf{February 1996 – February 1999}}] \leavevmode
Research registrar in neurology at the MRC Cyclotron Unit, Hammersmith
hospital and Oxford University, Department of Physiology

\end{description}


\section{Medical Positions}
\label{research/cv_wrapper:medical-positions}\begin{description}
\item[{\textbf{June 1995 – January 1996}}] \leavevmode
Registrar in neurology at the Radcliffe Infirmary, Oxford, UK

\item[{\textbf{August 1994 – April 1995}}] \leavevmode
Senior house officer in neurology at The National Hospital for Neurology, Queen square, London

\item[{\textbf{August 1992 – July 1994}}] \leavevmode
Senior house officer rotation at St Bartholomew’s Hospital, London

\item[{\textbf{February 1992 – July 1992}}] \leavevmode
Senior house officer in neurosciences at Addenbrookes Hospital, Cambridge

\item[{\textbf{August 1991 – January 1992}}] \leavevmode
Research worker at the Institute of Psychiatry, London

\item[{\textbf{August 1990 – July 1991}}] \leavevmode
House officer at the Royal London Hospital

\end{description}


\section{Education and Qualifications}
\label{research/cv_wrapper:education-and-qualifications}\begin{description}
\item[{\textbf{1994}}] \leavevmode
Membership of the Royal College of Physicians (UK)

\item[{\textbf{1987 – 1990}}] \leavevmode
Bachelor of Medicine and Surgery (MB BChir)

\item[{\textbf{1984 – 1987}}] \leavevmode
BA 2.i; Experimental Psychology, Cambridge University (UK)

\end{description}


\section{Awards}
\label{research/cv_wrapper:awards}\begin{description}
\item[{\textbf{1996}}] \leavevmode
British Brain and Spine Foundation 3 year research training fellowship

\item[{\textbf{1984}}] \leavevmode
Open Entrance Scholarship to Cambridge University

\end{description}


\section{Research Program}
\label{research/cv_wrapper:research-program}
I work on developing, implementing and teaching methods of functional data
analysis including FMRI and diffusion weighted imaging.  have a strong interest
in open-source software development as the basis of methods research and
understanding.  My primary focus is the development of a large international
collaboration for new imaging software development (neuroimaging.scipy.org).
Particular scientific interests include technical and theoretical problems in
spatial normalization (inter-subject registration), region of interest analysis,
and new methods of statistical analysis across subjects and tasks.


\section{Scientific activities}
\label{research/cv_wrapper:scientific-activities}
Reviewer for NeuroImage, Human Brain Mapping, Journal of Cognitive Neuroscience,
Neuroscience Letters, Clinical Neurophysiology, Journal of Neuroimaging,
Frontiers in Neuroinformatics


\subsection{Supervisor of two PhD students}
\label{research/cv_wrapper:supervisor-of-two-phd-students}\begin{description}
\item[{\textbf{2000 – 2004} \emph{Katja Osswald}}] \leavevmode
The role of SMA and basal ganglia in motor learning, mechanisms of apraxia
and methods of functional MRI analysis (submitted May 2004).  Katja is now a
clinical psychologist.

\item[{\textbf{2001 – 2004} \emph{Jessica Grahn}}] \leavevmode
The functional anatomy of musical beat perception. Jessica is currently an
assistant professor in the department of psychology in the university of
Western Ontario.

\end{description}


\subsection{Advisor to two post-doctoral researchers}
\label{research/cv_wrapper:advisor-to-two-post-doctoral-researchers}\begin{description}
\item[{\textbf{2001 – 2002} \emph{Alexandre Andrade}}] \leavevmode
working on surface-based FMRI statistics, coherence analysis.  Alexandre is
now professor in biophysics in Lisbon, Portugal

\item[{\textbf{2002 – 2006} \emph{Ferath Kherif}}] \leavevmode
working on multivariate statistics for clustering and diagnostics of
functional imaging data. Ferath is currently a prinicpal investigator at the
Service of Neurology, Centre Hospitalier Universitaire Vaudois in Lausanne,
Switzerland.

\end{description}


\subsection{Teaching and Tutorials}
\label{research/cv_wrapper:teaching-and-tutorials}
Author of a large number of imaging tutorial pages (see
\href{http://imaging.mrc-cbu.cam.ac.uk/imaging}{http://imaging.mrc-cbu.cam.ac.uk/imaging}).

Have given many invited talks on various topics in neuroimaging methods in
Cambridge, London, Oxford, York, Sheffield, Paris, Lyon, Marseille, Tokyo,
Buenos Aires, Berkeley, Stanford.
\begin{itemize}
\item {} 
2008 – : regular post-graduate teaching on the course on imaging organized by
Mark D'Esposito, and Neuroscience Seminar Series organized by Sonia Bishop

\item {} 
2004, 2006, 2007: Invited speaker for Human Brain Mapping conference course on FMRI

\item {} 
2005: (with Ansgar Furst) delivered 4 day SPM course in Oslo

\item {} 
2005: Faculty for Yale SPM course

\item {} 
2006 – 2008: Neuroscience supervisor for Jesus College, Cambridge.

\item {} 
2000 – 2003: Invited speaker at annual functional imaging courses held in Paris.

\item {} 
2001: Delivered a three day neuroimaging / SPM course in University of Melbourne.

\end{itemize}

\section{Articles and abstracts}
\label{research/cv_wrapper:publications}\label{research/cv_wrapper:articles-and-abstracts}

\begin{itemize}

\subsection{Computing}
\label{research/cv_wrapper:pubs-computing}\label{research/cv_wrapper:computing}

\subsubsection{Articles}
\label{research/cv_wrapper:articles}

\bibitem[Millman2007]{Millman2007}{\phantomsection\label{research/cv_wrapper:millman2007} 
K Jarrod Millman and Matthew Brett (2007) ``Analysis of functional
magnetic resonance imaging in Python''. \emph{Computing in Science \&
Engineering} , 2007. pp. 52--55.  
}

\subsubsection{Abstracts}
\label{research/cv_wrapper:abstracts}

\bibitem[Brett2009nipy]{Brett2009nipy}{\phantomsection\label{research/cv_wrapper:brett2009nipy} 
Matthew Brett, et al. (2009) ``NIPY: an open library and development
framework for FMRI data analysis''. \emph{NeuroImage} 47, 2009. pp. S196.  
}
\bibitem[Taylor2005brainpy]{Taylor2005brainpy}{\phantomsection\label{research/cv_wrapper:taylor2005brainpy} 
Jonathan E Taylor, et al. (2005) ``BrainPy: an open source environment
for the analysis and visualization of human brain data''. \emph{Neuroimage}
26, 2005. pp. 763.  \href{http://nipy.sourceforge.net/nipy/stable/references/brainpy\_abstract.html}{content}
}

\subsection{Methodology}
\label{research/cv_wrapper:methodology}\label{research/cv_wrapper:pubs-methodology}

\subsubsection{Articles and book chapters}
\label{research/cv_wrapper:articles-and-book-chapters}

\bibitem[Poline2012]{Poline2012}{\phantomsection\label{research/cv_wrapper:poline2012} 
Jean-Baptiste Poline and Matthew Brett (2012) ``The general linear
model and fMRI: Does love last forever?''. \emph{NeuroImage} 62, 2012. pp.
871--880.  \href{http://dx.doi.org/10.1016/j.neuroimage.2012.01.133}{doi: 10.1016/j.neuroimage.2012.01.133} 
}
\bibitem[Schwarzbauer2010]{Schwarzbauer2010}{\phantomsection\label{research/cv_wrapper:schwarzbauer2010} 
Christian Schwarzbauer, Toralf Mildner, Wolfgang Heinke and Matthew
Brett and Ralf Deichmann (2010) ``Dual echo EPI--the method of choice
for fMRI in the presence of magnetic field inhomogeneities?''.
\emph{Neuroimage} 49, Jan 2010. pp. 316--326.  \href{http://dx.doi.org/10.1016/j.neuroimage.2009.08.032}{doi: 10.1016/j.neuroimage.2009.08.032}
}
\bibitem[Poldrack2008]{Poldrack2008}{\phantomsection\label{research/cv_wrapper:poldrack2008} 
Russell A Poldrack, et al. (2008) ``Guidelines for reporting an fMRI
study.''. \emph{Neuroimage} 40, Apr 2008. pp. 409--414.  \href{http://dx.doi.org/10.1016/j.neuroimage.2007.11.048}{doi: 10.1016/j.neuroimage.2007.11.048}
}
\bibitem[Crinion2007]{Crinion2007}{\phantomsection\label{research/cv_wrapper:crinion2007} 
Jenny Crinion, et al. (2007) ``Spatial normalization of lesioned
brains: performance evaluation and impact on fMRI analyses.''.
\emph{Neuroimage} 37, Sep 2007. pp. 866--875.  \href{http://dx.doi.org/10.1016/j.neuroimage.2007.04.065}{doi: 10.1016/j.neuroimage.2007.04.065}
}
\bibitem[Aston2006]{Aston2006}{\phantomsection\label{research/cv_wrapper:aston2006} 
John A D Aston, Federico E Turkheimer and Matthew Brett (2006) ``HBM
functional imaging analysis contest data analysis in wavelet space.''.
\emph{Hum Brain Mapp} 27, May 2006. pp. 372--379.  \href{http://dx.doi.org/10.1002/hbm.20244}{doi: 10.1002/hbm.20244}
}
\bibitem[Saxe2006]{Saxe2006}{\phantomsection\label{research/cv_wrapper:saxe2006} 
Rebecca Saxe, Matthew Brett and Nancy Kanwisher (2006) ``Divide and
conquer: a defense of functional localizers.''. \emph{Neuroimage} 30, May
2006. pp. 1088--96; discussion 1097-9.  \href{http://dx.doi.org/10.1016/j.neuroimage.2005.12.062}{doi: 10.1016/j.neuroimage.2005.12.062}
}
\bibitem[Nichols2005]{Nichols2005}{\phantomsection\label{research/cv_wrapper:nichols2005} 
Thomas Nichols, Matthew Brett, Jesper Andersson, Tor Wager and
Jean-Baptiste Poline (2005) ``Valid conjunction inference with the
minimum statistic.''. \emph{Neuroimage} 25, Apr 2005. pp. 653--660.  \href{http://dx.doi.org/10.1016/j.neuroimage.2004.12.005}{doi: 10.1016/j.neuroimage.2004.12.005}
}
\bibitem[Brett2003]{Brett2003}{\phantomsection\label{research/cv_wrapper:brett2003} 
Matthew Brett, William D Penny and Stefan Kiebel (2004) ``Introduction
to random field theory''. In Richard S.J. Frackowiak and John T.
Ashburner and William D. Penny and Semir Zeki (Eds.) \emph{Human Brain
Function, Second Edition}
}
\bibitem[Cusack2003]{Cusack2003}{\phantomsection\label{research/cv_wrapper:cusack2003} 
Rhodri Cusack, Matthew Brett and Katja Osswald (2003) ``An evaluation
of the use of magnetic field maps to undistort echo-planar images.''.
\emph{Neuroimage} 18, Jan 2003. pp. 127--142.
}
\bibitem[Kherif2003]{Kherif2003}{\phantomsection\label{research/cv_wrapper:kherif2003} 
Ferath Kherif, Jean-Baptiste Poline and S (2003) ``Group analysis in
functional neuroimaging: selecting subjects using similarity
measures.''. \emph{Neuroimage} 20, Dec 2003. pp. 2197--2208.
}
\bibitem[Brett2002]{Brett2002}{\phantomsection\label{research/cv_wrapper:brett2002} 
Matthew Brett, Ingrid S Johnsrude and Adrian M Owen (2002) ``The
problem of functional localization in the human brain.''. \emph{Nat Rev
Neurosci} 3, Mar 2002. pp. 243--249.  \href{http://dx.doi.org/10.1038/nrn756}{doi: 10.1038/nrn756}
}
\bibitem[Hammers2002]{Hammers2002}{\phantomsection\label{research/cv_wrapper:hammers2002} 
Alexander Hammers, et al. (2002) ``Implementation and application of a
brain template for multiple volumes of interest.''. \emph{Hum Brain Mapp}
15, Mar 2002. pp. 165--174.
}
\bibitem[Brett2001]{Brett2001}{\phantomsection\label{research/cv_wrapper:brett2001} 
Matthew Brett, Alexander P Leff, Chris Rorden and John Ashburner
(2001) ``Spatial normalization of brain images with focal lesions
using cost function masking.''. \emph{Neuroimage} 14, Aug 2001. pp.
486--500.  \href{http://dx.doi.org/10.1006/nimg.2001.0845}{doi: 10.1006/nimg.2001.0845}
}
\bibitem[Gustard2001]{Gustard2001}{\phantomsection\label{research/cv_wrapper:gustard2001} 
Sharon Gustard, et al. (2001) ``Effect of slice orientation on
reproducibility of fMRI motor activation at 3 Tesla.''. \emph{Magn Reson
Imaging} 19, Dec 2001. pp. 1323--1331.
}
\bibitem[Turkheimer2001]{Turkheimer2001}{\phantomsection\label{research/cv_wrapper:turkheimer2001} 
Federico Turkheimer, Matthew Brett, D Visvikis and V J Cunningham
(2001) ``Statistical Estimation of PET Images in the Wavelet Domain''.
In Gjedde, A. (Eds.) \emph{Physiological imaging of the brain with PET}, :
}
\bibitem[Rorden2000]{Rorden2000}{\phantomsection\label{research/cv_wrapper:rorden2000} 
Chris Rorden and Matthew Brett (2000) ``Stereotaxic display of brain
lesions.''. \emph{Behav Neurol} 12, 2000. pp. 191--200.  
}
\bibitem[Turkheimer2000]{Turkheimer2000}{\phantomsection\label{research/cv_wrapper:turkheimer2000} 
Federico E Turkheimer, et al. (2000) ``Statistical modeling of
positron emission tomography images in wavelet space.''. \emph{J Cereb
Blood Flow Metab} 20, Nov 2000. pp. 1610--1618.  \href{http://dx.doi.org/10.1097/00004647-200011000-00011}{doi: 10.1097/00004647-200011000-00011}
}
\bibitem[Turkheimer1999]{Turkheimer1999}{\phantomsection\label{research/cv_wrapper:turkheimer1999} 
Federico E Turkheimer, Matthew Brett and D Visvikis and Vincent J
Cunningham (1999) ``Multiresolution analysis of emission tomography
images in the wavelet domain.''. \emph{J Cereb Blood Flow Metab} 19, Nov
1999. pp. 1189--1208.  \href{http://dx.doi.org/10.1097/00004647-199911000-00003}{doi: 10.1097/00004647-199911000-00003}
}

\subsubsection{Abstracts and conference papers}
\label{research/cv_wrapper:abstracts-and-conference-papers}

\bibitem[Garyfallidis2010hbm]{Garyfallidis2010hbm}{\phantomsection\label{research/cv_wrapper:garyfallidis2010hbm} 
Eleftherios Garyfallidis, Matthew Brett and Ian Nimmo-Smith (2010)
``Fast Dimensionality Reduction for Brain Tractography Clustering''.
\emph{NeuroImage} 47, 2010. pp. ?.
}
\bibitem[Garyfallidis2010ismrm]{Garyfallidis2010ismrm}{\phantomsection\label{research/cv_wrapper:garyfallidis2010ismrm} 
Eleftherios Garyfallidis, Matthew Brett, Tsiaras V, Vogiatzis G and
Ian Nimmo-Smith (2010) ``Identification of corresponding tracks in
diffusion MRI tractographies''. \emph{Proceedings of the International
Society for Magnetic Resonance in Medicine} 18, 2010. pp. ?.
}
\bibitem[Brett2003er]{Brett2003er}{\phantomsection\label{research/cv_wrapper:brett2003er} 
Matthew Brett, Ian Nimmo-Smith, Katja Osswald and Ed T Bullmore
(2003) ``Model fitting and power in fast event related designs''.
\emph{NeuroImage} 19, 2003. pp. abstract 791.
}
\bibitem[Brett2002marsbar]{Brett2002marsbar}{\phantomsection\label{research/cv_wrapper:brett2002marsbar} 
Matthew Brett, Jean-Luc Anton and Romain Valabregue and Jean-Baptiste
Poline (2002) ``Region of interest analysis using an SPM toolbox''.
\emph{Neuroimage} 16, 2002. pp. 1140--1141.  
}
\bibitem[Fadili2002]{Fadili2002}{\phantomsection\label{research/cv_wrapper:fadili2002} 
M Jalal Fadili, Ed T Bullmore and Matthew Brett 2002. Wavelet methods
for characterising mono-and poly-fractal noise structures in shortish
time series: an application to functional MRI.
}
\bibitem[Brett2001mni]{Brett2001mni}{\phantomsection\label{research/cv_wrapper:brett2001mni} 
Matthew Brett, Kalina Christoff, Rhodri Cusack and Jack Lancaster
(2001) ``Using the Talairach atlas with the MNI template''.
\emph{Neuroimage} 13, 2001. pp. S85.  
}
\bibitem[Johnsrude2001cyto]{Johnsrude2001cyto}{\phantomsection\label{research/cv_wrapper:johnsrude2001cyto} 
Ingrid S Johnsrude, et al. (2001) ``Cytoarchitectonic
region-of-interest analysis of auditory imaging data''. \emph{NeuroImage}
13, 2001. pp. S897.
}
\bibitem[Brett1999time]{Brett1999time}{\phantomsection\label{research/cv_wrapper:brett1999time} 
Matthew Brett, Peter Bloomfield, David J Brooks, John F Stein and
Paul Grasby (1999) ``Scan order effects in PET activation studies are
caused by motion artifact''. \emph{NeuroImage} 9, 1999. pp. S56.  
}

\subsection{Functional imaging of movement and thinking}
\label{research/cv_wrapper:functional-imaging-of-movement-and-thinking}\label{research/cv_wrapper:pubs-movement}

\subsubsection{Articles}
\label{research/cv_wrapper:id1}

\bibitem[Grahn2009]{Grahn2009}{\phantomsection\label{research/cv_wrapper:grahn2009} 
Jessica A Grahn and Matthew Brett (2009) ``Impairment of beat-based
rhythm discrimination in Parkinson's disease.''. \emph{Cortex} 45, Jan
2009. pp. 54--61.  \href{http://dx.doi.org/10.1016/j.cortex.2008.01.005}{doi: 10.1016/j.cortex.2008.01.005}
}
\bibitem[Grahn2007]{Grahn2007}{\phantomsection\label{research/cv_wrapper:grahn2007} 
Jessica A Grahn and Matthew Brett (2007) ``Rhythm and beat perception
in motor areas of the brain.''. \emph{J Cogn Neurosci} 19, May 2007. pp.
893--906.  \href{http://dx.doi.org/10.1162/jocn.2007.19.5.893}{doi: 10.1162/jocn.2007.19.5.893}
}
\bibitem[Spencer2007]{Spencer2007}{\phantomsection\label{research/cv_wrapper:spencer2007} 
Rebecca M C Spencer, Timothy Verstynen, Matthew Brett and Richard
Ivry (2007) ``Cerebellar activation during discrete and not continuous
timed movements: an fMRI study.''. \emph{Neuroimage} 36, Jun 2007. pp.
378--387.  \href{http://dx.doi.org/10.1016/j.neuroimage.2007.03.009}{doi: 10.1016/j.neuroimage.2007.03.009}
}
\bibitem[Dove2006]{Dove2006}{\phantomsection\label{research/cv_wrapper:dove2006} 
Anja Dove, Matthew Brett, Rhodri Cusack and Adrian M Owen (2006)
``Dissociable contributions of the mid-ventrolateral frontal cortex
and the medial temporal lobe system to human memory.''. \emph{Neuroimage}
31, Jul 2006. pp. 1790--1801.  \href{http://dx.doi.org/10.1016/j.neuroimage.2006.02.035}{doi: 10.1016/j.neuroimage.2006.02.035}
}
\bibitem[Bishop2004]{Bishop2004}{\phantomsection\label{research/cv_wrapper:bishop2004} 
Sonia Bishop, John Duncan, Matthew Brett and Andrew D Lawrence (2004)
``Prefrontal cortical function and anxiety: controlling attention to
threat-related stimuli.''. \emph{Nat Neurosci} 7, Feb 2004. pp. 184--188.  \href{http://dx.doi.org/10.1038/nn1173}{doi: 10.1038/nn1173}
}
\bibitem[Graham2003]{Graham2003}{\phantomsection\label{research/cv_wrapper:graham2003} 
Kim S Graham, Andy C H Lee, Matthew Brett and Karalyn Patterson
(2003) ``The neural basis of autobiographical and semantic memory: new
evidence from three PET studies.''. \emph{Cogn Affect Behav Neurosci} 3,
Sep 2003. pp. 234--254.
}
\bibitem[Kellenbach2003]{Kellenbach2003}{\phantomsection\label{research/cv_wrapper:kellenbach2003} 
Marion L Kellenbach, Matthew Brett and Karalyn Patterson (2003)
``Actions speak louder than functions: the importance of
manipulability and action in tool representation.''. \emph{J Cogn Neurosci}
15, Jan 2003. pp. 30--46.  \href{http://dx.doi.org/10.1162/089892903321107800}{doi: 10.1162/089892903321107800} 
}
\bibitem[Kellenbach2001]{Kellenbach2001}{\phantomsection\label{research/cv_wrapper:kellenbach2001} 
Marion L Kellenbach, Matthew Brett and Karalyn Patterson (2001)
``Large, colorful, or noisy? Attribute- and modality-specific
activations during retrieval of perceptual attribute knowledge.''.
\emph{Cogn Affect Behav Neurosci} 1, Sep 2001. pp. 207--221.  
}

\subsubsection{Abstracts and conference papers}
\label{research/cv_wrapper:id2}

\bibitem[Osswald2002]{Osswald2002}{\phantomsection\label{research/cv_wrapper:osswald2002} 
Katja Osswald, John Duncan, Gordon D Logan and Matthew Brett (2002)
``Automatic response selection -- functional imaging of practice
effects''. \emph{Abstract Viewer/Itinerary Planner. Washington, DC: Society
for Neuroscience. Program No. 163.2. Online} , 2002. pp. .  
}
\bibitem[Dove2001encoding]{Dove2001encoding}{\phantomsection\label{research/cv_wrapper:dove2001encoding} 
Anje Dove, James B Rowe, Matthew Brett and Adrian M Owen (2001)
``Neural correlates of passive and active encoding and retrieval: A 3T
fMRI study''. \emph{NeuroImage} 13, 2001. pp. S660.
}
\bibitem[Brett1998pmc]{Brett1998pmc}{\phantomsection\label{research/cv_wrapper:brett1998pmc} 
Matthew Brett, John F Stein and David J Brooks (1998) ``The role of
the lateral premotor cortex in conditional and imitated praxis''.
\emph{NeuroImage} 9, 1998. pp. S987.  
}
\bibitem[Brett1997sma]{Brett1997sma}{\phantomsection\label{research/cv_wrapper:brett1997sma} 
Matthew Brett, I Harry Jenkins, John F Stein and David J Brooks
(1997) ``Movement selection without preparation does not activate the
SMA''. \emph{NeuroImage} 5, 1997. pp. S269.  
}

\subsection{About other things}
\label{research/cv_wrapper:about-other-things}\label{research/cv_wrapper:pubs-other}

\bibitem[Brett2001a]{Brett2001a}{\phantomsection\label{research/cv_wrapper:brett2001a} 
Matthew Brett and Sallie Baxendale (2001) ``Motherhood and memory: a
review.''. \emph{Psychoneuroendocrinology} 26, May 2001. pp. 339--362.  
}
\bibitem[Brett1999]{Brett1999}{\phantomsection\label{research/cv_wrapper:brett1999} 
M Brett, et al. (1999) ``Transthyretin Leu12Pro is associated with
systemic, neuropathic and leptomeningeal amyloidosis.''. \emph{Brain} 122 (
Pt 2), Feb 1999. pp. 183--190.  
}
\bibitem[Brett1995]{Brett1995}{\phantomsection\label{research/cv_wrapper:brett1995} 
M Brett, R Greenwood, J Powell and P Morris (1995) ``Late functional
recovery with a novel community rehabilitation programme after herpes
simplex encephalitis''. \emph{Clinical Rehabilitation} 9, Aug 1995. pp.
267--270.  \href{http://dx.doi.org/10.1177/026921559500900315}{doi: 10.1177/026921559500900315}
}
\bibitem[Chesser1989]{Chesser1989}{\phantomsection\label{research/cv_wrapper:chesser1989} 
Alistair Chesser and Matthew Brett 1989. Clinical teaching in
context: a factor analysis of student ratings. Research in Medical
Education Conference (RIME).
}

\end{itemize}

\end{document}
